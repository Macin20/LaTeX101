% 这是一个练习,自己亲自把这篇文字排版出来。

\documentclass[UTF8]{ctexart}
\newtheorem{them}{定理}
\usepackage{geometry}
\geometry{a6paper,centering,scale=0.8}

\usepackage[nottoc]{tocbibind}

\title{杂谈勾股定理}
\author{陈诗}
\date{\today}

\bibliographystyle{plain} %规定引用的样式

\begin{document}
    \maketitle % 输出标题
    \tableofcontents %输出目录。因为有目录,所以需要编译两次。

    \section{勾股定理在古代}
    公元前十一世纪,数学家商高(西周初年人)就提出“勾三、股四、弦五”。编写于公元前一世纪以前的《周髀算经》中记录着商高与周公的一段对话。
    商高说:
    \begin{quote}
    ……故折矩,勾广三,股修四,经隅五。  
    \end{quote}
    意为:当直角三角形的两条直角边分别为3(勾)和4(股)时,径隅(弦)则为5。
    
    % 使用空行进行分段。空行只分段,不会增加行间距。
    以后人们就简单地把这个事实说成“勾三股四弦五”,根据该典故称勾股定理为商高定理。
公元三世纪,三国时代的赵爽对《周髀算经》内的勾股定理作出了详细注释,记录于《九章算术》中“勾股各自乘,并而开方除之,即弦”,赵爽创制了一幅“勾股圆方图”,用形数结合得到方法,给出了勾股定理的详细证明。
    
    在中国清朝末年,数学家华蘅芳提出了二十多种对于勾股定理证法。

    远在公元前约三千年的古巴比伦人就知道和应用勾股定理,他们还知道许多勾股数组。美国哥伦比亚大学图书馆内收藏着一块编号为“普林顿322”的古巴比伦泥板,上面就记载了很多勾股数。古埃及人在建筑宏伟的金字塔和测量尼罗河泛滥后的土地时,也应用过勾股定理。

    公元前六世纪,希腊数学家毕达哥拉斯证明了勾股定理,因而西方人都习惯地称这个定理为毕达哥拉斯定理。

    公元前4世纪,希腊数学家欧几里得\footnote{欧几里德,约公元前 330--275年。}在《几何原本》(第Ⅰ卷,命题47)中给出一个证明。
    \section{勾股定理在近代}
    勾股定理用现代的语言可以表述为:

    \begin{them}
        直角三角形斜边的平方等于两腰的平方和。
    \end{them}

    可以将上述定理转化为符号语言:设直角三角形 $ABC$ ,其中 $ \angle C = 90  $,则有

    \begin{equation}
        AB^2 = BC^2 + AC^2
    \end{equation}

    满足上述公式的整数被称为勾股数。

    \bibliography{math} %提示TEX从文献数据库math中获取文献信息,打印参考文献列表


\end{document}
